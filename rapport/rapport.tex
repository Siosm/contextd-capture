\documentclass[pdftex,a4paper,titlepage,11pt,openright]{article}
\usepackage[T1]{fontenc}
\usepackage[utf8]{inputenc}
\usepackage[english,francais]{babel}
\usepackage{listings}
\usepackage{setspace}

\usepackage{avant}
\usepackage{fancyvrb}
\usepackage{fancyhdr}
\pagestyle{fancy}
\fancyhf{}
\fancyhead[LE,RO]{\itshape\thepage}
% \fancyhead[LO]{\itshape\rightmark}
% \fancyhead[RE]{\itshape\leftmark}
\renewcommand{\headrulewidth}{0.5pt}
\addtolength{\headheight}{0.5pt}
\renewcommand\footrulewidth{0pt}
\fancypagestyle{plain}{
    \fancyhead{}
    \renewcommand{\headrulewidth}{0pt}}
\usepackage{textcomp}
\usepackage{relsize}
\usepackage{amssymb}
\usepackage[colorlinks=true,linkcolor=black,citecolor=black,urlcolor=black,filecolor=black]{hyperref}
\usepackage{framed}
\usepackage{geometry}
\usepackage[pdftex]{graphicx}
\usepackage{makeidx}
\addtolength{\textwidth}{1cm}
\setlength{\textheight}{24cm} 	% Hauteur de la zone de texte


% nouvelle commande pour un joli nom
\newcommand{\nom}[1]{\textsc{#1}}

% commande pour une zolie ligne
\newcommand{\ligne}[1][1pt]{
  \par\noindent
  \rule[.5ex]{\linewidth}{#1}\par}

% nettoyer une page blanche avant une page de chapitre en mode openright
\newcommand{\clearemptydoublepage}{
	\newpage{\pagestyle{empty}\cleardoublepage}}


\makeindex

\begin{document}

% augmenter l'espacement entre plusieurs paragraphes plutôt que de passer des lignes quand il faut pas
\setlength{\parskip}{2.4ex}

\title{\ligne{\Large}\textbf{Project de deuxième année}\\
\Large LSM ? PGST ?} %TODO
\author{\nom{Dimitri Gressin} \& \nom{Timothée Ravier}\\\\\nom{Pilote : Jérémy Briffaut}}
\date{xx xxx 201x} %TODO


% titre
\maketitle

% page blanche
\clearemptydoublepage

% table des matières
\setcounter{secnumdepth}{2}
\setcounter{tocdepth}{2}
\addtocontents{toc}{\protect\thispagestyle{empty}}
\tableofcontents
\addtocounter{page}{-1}

\newpage

\section*{Introduction} \addcontentsline{toc}{section}{Introduction}
Ce rapport présente le travail et les résultats obtenus après x mois dans le cadre de notre projet d'application de deuxième année. %TODO

~

Le but de ce projet est %TODO

~

 %TODO

\newpage

\section{Systemtap}
Nous avons débuter avec Systemtap qui est %TODO

Il fallait obtenir :
\begin{itemize}
	\item la liste des fichiers créés, ouverts, modifiés par l'ensemble des processus, et les contextes selinux associés
	\item la liste des connexions ouvertes par le systeme (socket) et faire correspondre l'adresse ip avec le nom de domaine pour obtenir une url
\end{itemize}

Nous avons partiellement atteint ces critères mais nous nous sommes rendu compte d'une particularité de l'implémentation de Systemtap qui ne corespondait pas avec notre besoin final. Nous nous sommes heurté à la simplicité de Systemtap qui vise un apprentissage rapide pour un usage ciblé.

De plus, les instructions décrites dans un script écrit pour Systemtap sont exécutées après que le code de l'appel system correspondant ai été exécuté.

%\includegraphics[scale=0.4]{stap_internals.png} TODO

Cela ne correspond donc pas au besoin de sécurité énoncé précédement.


\newpage


\section{Linux Security Modules}

Nous avons donc décider avec l'accord de Jérémy Briffaut de nous tourner vers les ``Linux Security Modules''.

\newpage


\section{Résultats obtenus}

\subsection{Graphes simples}
Les graphes ne contenant que peu de noeuds (moins de 80) sont traités rapidement et donnent des résultats probants, même si le faible nombre de noeud ne permet pas une simplification poussée. Voici donc les résultats que l'on peut obtenir :



\subsection{Graphe réel}
Nous présentons dans cette section les résultats obtenus lors des tests sur un graphe réel de politique PIGA. Il faut noter que pour limiter la durée des traitements nous n'avons pas tenu compte ni des noms des arrêtes ni de l'orientation de celles-ci. Nous importons alors un graphe de 27Mo en 8 secondes environ.


\newpage


\clearemptydoublepage

\section*{Conclusion} \addcontentsline{toc}{section}{Conclusion}

Ce projet aura été pour nous l'occasion d'apprendre un langage de programmation orienté objets, qui nous a permis de gagner en efficacité lors de la phase de développement. En revanche, le fait que nous découvrions le C++ a limité nos possibilités d'optimisations. Le temps de traitement d'un graphe contenant une grande quantité de noeud étant élevé, il paraît nécessaire de paralléliser les algorithmes que nous avons appliqués, pour tirer pleinement partie des machines multi-coeurs ou multi-processeurs, par exemple en utilisant les Intel Threading Building Blocks \cite{TBB}.

De plus, bien que vérifiée, notre implémentation de l'algorithme de décomposition modulaire peut contenir des bugs. Il faut donc prévoir un outils permettant la vérification des résultats obtenus par cette méthode avec ceux obtenus par le parcours simple de graphes. Il sera donc toujours nécessaire de produire l'intégralité des chemins entre deux noeuds à partir du graphe d'origine pour s'assurer que les graphes réduits sont bien justes.

Il faut noter, au vu des résultats, que la décomposition modulaire de graphes permettra fort probablement non seulement d'accélerer le parcours de graphes une fois l'intégralité des chemins générés mais aussi de réduire la taille finale de l'ensemble formé par tous les chemins.

Enfin, notre programme se limite à la décomposition de graphes orientés. Les algorithmes diffèrent de ceux applicables aux graphes non-orientés et leur mise en oeuvre nécessiterai la réécriture d'une grande partie du code pour assurer la rapidité et l'efficacité de notre implémentation.


\newpage
\addcontentsline{toc}{section}{Annexes}
\addcontentsline{toc}{subsection}{Remarques}
\addcontentsline{toc}{subsection}{Liens et références}
\subsection*{Remarques}

Sous Mac OS X, les sauts de lignes (\textbackslash n) sont parfois accompagnés ou remplacés par des ``carriage return'' (\textbackslash r). Notre programme ne permet pas de gérer ce cas particulier; il faut donc s'assurer que seuls les ``line feed'' (\textbackslash n) sont présents.

\subsection*{Liens et références}
\begin{thebibliography}{40}
\bibitem{TBB} \textit{Intel Threading Building Blocks}, \url{http://www.threadingbuildingblocks.org/}

\bibitem{HDR} \textit{Aspects Algorithmiques de la décomposition modulaire, par Christophe Paul}, \url{http://www.lirmm.fr/~paul/HdR/hdr.pdf}

\bibitem{SOURCE} Code source disponible sur le serveur de projet STI, \url{http://projetsti.ensi-bourges.fr/projects/2012-p4-smg}.
\end{thebibliography}

%\printindex

\end{document}
