\section*{Introduction} \addcontentsline{toc}{section}{Introduction}
Ce rapport présente le travail et les résultats obtenus dans le cadre de notre projet d'application de deuxième année. Ce projet s'inscrit dans le cadre des projets de recherche menés au Laboratoire d'Informatique Fondamentale d'Orléans (LIFO) par l'équipe Sécurité et Distribution des Systèmes (SDS) sur la création d'un système d'exploitation sécurisé basé sur Linux.
% FIXME Est-ce correct ?

~

Ce projet a pour but la simplification et la généralisation du contrôle d'application par le démon contextd. Contextd est un démon résident en espace utilisateur qui commande et coordonne différents systèmes de sécurité (SELinux, PIGA-MAC, iptables...). Il permet de changer dynamiquement la configuration de ces outils de sécurité pour assurer la sécurité et la cohérence globale d'un système. Il constitue une mise en oeuvre avancée du principe de séparation des privilèges, limitant les droits des applications contrôlées au strict minimum, à chaque instant donné.

~

Mais pour fonctionner, Contextd doit avoir connaissance des actions entreprisent par chacune des applications que l'on veut surveiller. Jusqu'à présent, la communication entre les applications et contextd s'effectuait via un pluging ou un patch propre à chaque application. Contextd recueillait alors les demandes de chacunes de ces applications (lecture/écriture de fichiers, création de socket...) pour en déduire un domaine (web, ecommerce, mail...). Contextd était donc limité aux informations fournies par ces applications.

~

L'idée retenue consiste à déplacer la contrainte de communication au niveau du noyau, qui par l'intermédiaire des appels système a connaissance des actions effectuées par les programmes. Il s'agit donc de modifier le fonctionnement du noyau Linux ainsi que le comportement de contextd et la nature de ses sources d'information.

~