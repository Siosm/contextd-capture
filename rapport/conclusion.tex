\section*{Conclusion} \addcontentsline{toc}{section}{Conclusion}

Nous avons donc bien atteint les premiers objectifs consistants à récupérer des informations essentielles à contextd.

Le retard, sur la partie implémentation noyau, par rapport à notre planification est principalement dû à notre découverte très progressive des capacités offertes aux développeurs. Le livre Linux Kernel Development \cite{LKDSE} nous a permis de faire un bon en avant dans la compréhension du fonctionnement du noyau et notamment l'implémentation des appels système.

Malheureusement, seule une petite partie du projet est visible, du fait du retard pris au départ, avec systemtap, et dû au temps de compilation du noyau. Mais l'évolution actuelle du projet nous permettra d'atteindre les objectifs à la fin de l'année.

\newpage
\addcontentsline{toc}{section}{Annexes}
\addcontentsline{toc}{subsection}{Liens et références}

% \subsection*{Liens et références}
\begin{thebibliography}{40}
\bibitem{IBMRBST} \textit{IBM Redbooks : SystemTap: Instrumenting the Linux Kernel for Analyzing Performance and Functional Problems}, \url{http://www.redbooks.ibm.com/abstracts/redp4469.html}

\bibitem{LSMINTRO} \textit{Linux Security Modules : General Security Support for the Linux Kernel}, \url{http://citeseerx.ist.psu.edu/viewdoc/download?doi=10.1.1.84.6867&rep=rep1&type=pdf}

\bibitem{SOURCE} Code source (kernel 2.6.32 hardened r22, piga-systrans, et scripts systemtap) disponible sur le serveur de projet STI (le projet s'appelle Contextd Capture), \url{http://projetsti.ensi-bourges.fr/projects/promo2012-systemtap}.

\bibitem{LKDSE} \textit{Linux Kernel Development, Second Edition}, Robert Love, Novell Press
\bibitem{LKDSE} \textit{Linux Kernel Development, Third Edition}, Robert Love, Pearson Education, Inc.
\end{thebibliography}

%\printindex