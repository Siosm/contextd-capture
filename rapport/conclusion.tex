\section*{Conclusion} \addcontentsline{toc}{section}{Conclusion}

Le but de ce projet était de simplifier l'ajout d'application à PIGA-SYSTRANS et au système PIGA-OS. Cet objectif est atteint car nos travaux permettent, par exemple, de faire fonctionner OpenOffice.org/LibreOffice avec contextd, sans modifier ces applications.

De plus, l'aspect totalement générique de notre solution nous assure un fort potentiel de réutilisation. En effet, les ajouts dans contextd permmettant le chargement dynamique de programmes à contrôler nous permettent de se concentrer sur l'écriture de règles contextd, et non sur la modification des applications. Il ne reste alors plus qu'à modifier uniquement les applications possédant un comportment s'étendant sur plusieurs domaines et dont l'état interne nécessite des modifications lors d'un changement de domaine.

TODO Remarques sur SELinux / intégration avec PIGA-OS ...

% Le retard, sur la partie implémentation noyau, par rapport à notre planification est principalement dû à notre découverte très progressive des capacités offertes aux développeurs. Le livre Linux Kernel Development \cite{LKDTE} nous a permis de faire un bon en avant dans la compréhension du fonctionnement du noyau et notamment l'implémentation des appels système.

\newpage
\addcontentsline{toc}{section}{Annexes}
\addcontentsline{toc}{subsection}{Liens et références}

% \subsection*{Liens et références}
\begin{thebibliography}{40}

\bibitem{IBMRBST} \textit{IBM Redbooks : SystemTap: Instrumenting the Linux Kernel for Analyzing Performance and Functional Problems}, Bart Jacob, Paul Larson, Breno Henrique Leitao, Saulo Augusto M Martins da Silva, \url{http://www.redbooks.ibm.com/abstracts/redp4469.html}

\bibitem{LSMINTRO} \textit{Linux Security Modules : General Security Support for the Linux Kernel}, Chris Wright, Crispin Cowan, Stephen Smalley, James Morris, Greg Kroah-Hartman, \url{http://citeseerx.ist.psu.edu/viewdoc/download?doi=10.1.1.84.6867&rep=rep1&type=pdf}

\bibitem{LKDSE} \textit{Linux Kernel Development, Second Edition}, Robert Love, Novell Press
\bibitem{LKDTE} \textit{Linux Kernel Development, Third Edition}, Robert Love, Pearson Education, Inc.

\bibitem{MRHEL5} \textit{Managing Red Hat Enterprise Linux 5}, Daniel J Walsh, Karl MacMillan, \url{http://people.redhat.com/dwalsh/SELinux/Presentations/ManageRHEL5.pdf}

\bibitem{WCS} \textit{Wikipedia : Computer Security}, \url{http://en.wikipedia.org/wiki/Computer_security}

\bibitem{SOURCE} Code source (kernel 2.6.32 hardened r22, piga-systrans, et scripts systemtap) disponible sur le serveur de projet STI (le projet s'appelle Contextd Capture), \url{http://projetsti.ensi-bourges.fr/projects/promo2012-systemtap}.

\end{thebibliography}

%\printindex