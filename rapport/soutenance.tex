%%%%%%%%%%%%%%%%%%%%%%%%%%%%%%%%%%%%%%%%%%%%%%%%%%%%%%%%%%%%%%%%%%%%%%%%%%%%%%%%
% Soutenance de projet - Fichier modèle pour présentations Beamer
%%%%%%%%%%%%%%%%%%%%%%%%%%%%%%%%%%%%%%%%%%%%%%%%%%%%%%%%%%%%%%%%%%%%%%%%%%%%%%%%
\documentclass{beamer}
\usepackage[utf8]{inputenc}
\usepackage[francais]{babel}
\usepackage[T1]{fontenc}
\usepackage{textcomp}
\usepackage{relsize}
\usepackage{amssymb}
\usepackage{framed}


%%% THÈME TORINO AVEC MINIFRAMES MODIFIÉES - NÉCESSITE FICHIERS SPÉCIAUX
\usetheme[pageofpages=sur,% String used between the current page and the
                         % total page count.
          alternativetitlepage=true,% Use the fancy title page.
          titlepagelogo=logo,% Logo for the first page.
          titleline=true,
          watermark=watermark,% Watermark used in every page.
          watermarkheight=100px,% Height of the watermark.
          watermarkheightmult=4,% The watermark image is 4 times bigger
                                % than watermarkheight.
          ]{Torino}
\useoutertheme[subsection=true]{miniframes2}
\usecolortheme{freewilly}
%%% FIN DU SECOND THÈME

% nouvelle commande pour un joli nom
\newcommand{\nom}[1]{\textsc{#1}}

% commande pour une zolie ligne
\newcommand{\ligne}[1][1pt]{
  \par\noindent
  \rule[.5ex]{\linewidth}{#1}\par}

% \0
\newcommand{\slz}{$\backslash0$}

% commande pour un message réseau
\newcommand{\netmessage}[5]{\small \begin{framed}
\texttt{Message} \emph{#1} \texttt{- #2 - #3 $\rightarrow$ #4}\\\noindent
\rule{\linewidth}{0.4pt}

\footnotesize\noindent\texttt{#5}

\end{framed}}

%%% TITRE DE PAGE
\title{Simplification modulaire de graphe}
\author{Nicolas~Cornu \and Timothée~Ravier}
\institute{ENSI de Bourges}
\date{\today}

%%% POUR AVOIR UN PLAN QUI S'AFFICHE QUAND ON CHANGE DE SOUS-SECTION
\AtBeginSection[ ]
{
 \begin{frame}<beamer>
   \frametitle{Plan}
   \tableofcontents[currentsection]
  \end{frame}
}
\NoAutoSpaceBeforeFDP

\begin{document}

%%% LA PAGE DE TITRE, ON PEUT Y APPLIQUER DES OPTIONS COMME INSTITUTE CI-DESSOUS
{
	\framenumberoff
	\watermarkoff
	\institute{} % Vire le champ institut sur cette page
	\begin{frame}
	\titlepage
	\end{frame}
}


\section{Principe}
\subsection{Introduction}
\begin{frame}
\frametitle{Introduction}
\textbf{Simplification modulaire} : Regrouper ensemble les noeuds ayant des propriétés similaires.\\
\textbf{But} : Simplifier les graphes pour travailler dessus.\\
\textbf{Application} : Calculer l'ensemble des chemins entre deux noeuds.\\
\end{frame}

\subsection{Fonctionnalités}
\begin{frame}
\frametitle{Fonctionnalités}
\begin{itemize}
\item Importation de graphes depuis un fichier ``PIGA'' (.pol)
\item Calcul de l'arbre modulaire
\item Calcul du graphe groupé
\item Calcul du graphe simplifié en précisant les noeuds
\item Exportation en fichier PIGA (.pol) ou Graphviz
\end{itemize}
\end{frame}

\subsection{Algorithme}
\begin{frame}
\frametitle{Algorithme}
\begin{center}
	\includegraphics[scale=0.2]{diag.png}
\end{center}
\end{frame}

\section{Résultats}
\subsection{Graphes simples}
\begin{frame}
\frametitle{Graphes simples}
Fonctionne très bien avec de petits graphes.\\
\includegraphics[scale=0.3]{test-1.png}\hfill{}
\includegraphics[scale=0.45]{test-links.png}

\end{frame}
\subsection{Graphe réel}
\begin{frame}
\frametitle{Graphe réel}
\begin{itemize}
	\item Impossible de faire le graphe entier en un temps raisonnable\\
	\item \textbf{Causes}
		\begin{itemize}
		\item Trop de noeud (1704)
		\item Algorithmes pas assez optimisés
		\end{itemize}
\end{itemize}
\end{frame}


\begin{frame}
Nous avons donc travaillé sur moins de noeuds.
\begin{center}
\begin{tabular}{|c|c|}
	\hline
	Nombre de noeuds & Temps (sec.) \\
	\hline
	10 & 0,03 \\
	\hline
	20 & 0,3\\
	\hline
	30 & 1\\
	\hline
	40 & 2\\
	\hline
	50 & 4\\
	\hline
	70 & 34\\
	\hline
	80 & 180\\
	\hline
	100 & 630\\
	\hline
	120 & 1290 \\
	\hline
	150 & 3960\\
	\hline
\end{tabular}
\end{center}
\textbf{Et pour 1704 noeuds ?}
\end{frame}

\section{Conclusion}
\begin{frame}
\frametitle{Conclusion}
\textbf{Limitations}
\begin{itemize}
\item Ne gêre que les graphes non-orientés.
\item Vitesse (multi-thread).
\end{itemize}

\end{frame}


\end{document}
